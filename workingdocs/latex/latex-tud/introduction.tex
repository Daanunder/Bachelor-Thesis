\chapter{Introduction}\label{introduction}

Due to temperature and salinity differences density driven currents can
occur. Because important water systems such as the drinking water supply
and large coastal systems in the Netherlands are strongly influenced by
these currents it is important to understand these phenomena. With the
use of D-Flow FM, a software package by Deltares, the mixture of salt-
and fresh water can be numerically approximated. However, a side effect
of these numerical flow models is the occurrence of numerical diffusion
and dispersion caused by characteristics of the numerical discretization
scheme that is used.

Numerical diffusion is sometimes referred to as ``numerical viscosity''
since the associated approximation errors mimic the effect of an
increase in viscosity, i.e.~the solution is overdamped. Since viscous
properties of a fluid decrease the amplitude of the diffusivity rate in
advection-diffusion flow problems. Additionally, numerical dispersion is
related to unrealistic oscillations in an approximation of an
advection-diffusion problem that may occur if stability of the solution
is not ensured. Because avoiding such errors requires contrasting
measures a quantification of their responses to certain modelling
parameters is desirable.
\citep{zijlema_computational_2015, obrien_study_1950}

In order to quantify the numerical diffusion and dispersion terms in the
D-Flow FM model, and thus get an idea of the model's sensitivity to
certain parameters, a lock-exchange experiment will be set up which
serve as the basis for a sensitivity analysis. After an initial
qualitative analysis of the terms and settings found during an
explorative setup of the basic D-Flow FM model, a stable basis for the
parameter variations will be defined. In this basic model the desired
boundary conditions, the lenght and timescale of the simulation and the
constant parameters and modelling characteristics will be determined.
Herafter the creation of such basic model will be automated so that the
data required for the sensitivity analysis may be generated.

The required data consist of the numerical errors produced while
changing the grid resolution, the timestep size and parameters related
to the flow model. Subsequently the sensitivity S of a parameter P is
defined as the relative change of a state variable per change of this
parameter: S = (δx/x)/(δP/P). The state parameters are defined as the
difference between the approximation and the analytical solution at
predefined monitoring points, also known as the accuracy of the
approximation. Prematurely three state parameters are defined; the flow
velocity error, the density error and wave speed error.

During the explorative model setup and the data generation phase, the
observed data is analyzed and compared to certain indicator parameters
in order to explain the observed phenomena. These indicators have
previously been found to characterize the significance of certain
modelling parameters \citep{Shin2004} of a discretization scheme and the
associated error or are composites of important model parameter. Finally
the leverage of parameters, forcing functions and submodules of the
model are assessed based on the observed sensitivity for which a
possible explanation in terms of the indicators and model
characteristics will be sought.

This document describes the research questions and hypotheses in the
first chapter, subsequently it will describe the research strategy
explained above in further detail in chapter two. Lastly, in the
appendix a planning for this research project can be found.
