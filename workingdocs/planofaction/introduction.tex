\section{Introduction}\label{introduction}

Due to temperature and salinity differences density driven currents can
occur. Because important water systems such as the drinking water supply
and large coastal systems in the Netherlands are strongly influenced by
these currents it is important to understand these phenomena. With the
use of D-Flow FM, a software package by Deltares, the mixture of salt-
and fresh water can be numerically approximated. However, a side effect
of these numerical flow models is the occurrence of numerical diffusion
and dispersion caused by characteristics of the numerical discretization
scheme that is used.

Numerical diffusion is sometimes referred to as ``numerical viscosity''
since the produced error behaves as an increase in viscosity. Since
viscosity decreases the amplitude of the diffusivity rate, numerical
diffusion will increase this (amplitude) damping even further.
Additionally, numerical dispersion is an effect of the discretization of
higher order (non-linear) terms related to the hydrological dispersion
relation, which include discretization schemes to deal with advection
processes. Errors of this kind are often expressed through physically
unrealistic oscillations in the approximation. Because avoiding such
errors requires contrasting measures a quantification of their responses
to certain modelling parameters is desirable. \citep[
\citet{obrien_study_1950}]{zijlema_computational_2015}

In order to quantify the numerical diffusion and dispersion terms in the
D-Flow FM model, and thus get an idea of the model's sensitivity to
certain parameters, a lock-exchange experiment will be set up which will
be used as the basis for a sensitivity analysis. After an initial
qualitative analysis of the terms and settings found during this
explorative setup of the basic D-Flow FM model, the creation of such
basic model will be automated. From hereon the data required for the
sensitivity analysis will be generated.

The required data consist of the numerical errors produced while
changing the grid resolution, the timestep size and parameters related
to the numerical discretization scheme. The sensitivity S of a parameter
P is defined as the relative change of a state variable per change of
this parameter: S = (δx/x)/(δP/P). Three state parameters are defined as
the difference between the approximation and the analytical solution at
predefined monitoring points, also known as the accuracy of the
approximation; the flow velocity error, the density error and wave speed
error.

During the explorative model setup and the data generation phase, the
observed data is analyzed and compared to certain indicator parameters
that have previously been found characterize the significance of certain
modelling parameters \citep{Shin2004} of a characterize certain
discretization schemes and corresponding truncation errors in order to
explain the observed errors. Examples of such parameters are the
dimensionless Reynolds and Froude numbers or the ratio between the net
diffusivity coefficient and viscosity. Additionally the leverage of
parameters, forcing functions and submodules of the model are assessed
based on the observed sensitivity.
