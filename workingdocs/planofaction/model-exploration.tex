\chapter{Model Exploration}\label{model-exploration}

In order to provide a solid basis for the sensitivity analysis the basic
model has to be applicable to the research objective and sufficiently
robust to enable all parameters to be explored. To ensure a working
model first most default settings will be used to set up the
lock-exchange model, this will already require a minimal definition file
where general settings, physics, geometry, flow and numerical parameters
are set.

Thereafter possibly more realistic physical properties, boundary
conditions, a more relevant geometry and specifics of the flow model
such as bed friction and the will be defined. This should be done with
the research objective in mind thus possibly requires setting up three
different models for three different mesh grids in order to enable
stable testing, at the cost of consistency between models. These
trade-offs with respect to the research applicability and
generalizability will have to be made.

Further on the post-processing of the generated model will be performed,
in this way the results of the basic model can be qualitatively assessed
with respect to basic indicators (see \citet{indicator-parameters}).
Finally, monitoring points have to be set up at logically sound
locations and the basic models and subsequent post-processing have to be
automated as much as possible.

In short the following parts of the model should be explored: - General
settings

\begin{verbatim}
- Physics

- Geometry

- Flow

- Numerics

- Post-processing

- Automation
\end{verbatim}

\section{Boundary conditions}\label{boundary-conditions}

Choosing the correct boundary condition for the lock-exchange model may
be of significant importance considering the different boundary
conditions that can be chosen within D-Flow. These boundary conditions
can be divided into three groups and can all be related to the
lock-exchange quite well: - Boundary conditions that complement the
governing equations; continuity and momentum conservation. -
Supplementary boundary conditions that impose additional constraints at
a boundary. E.g. the weir at t=0 and the sidewalls and possibly the free
surface \citep{Adduce2012}. - Boundary conditions for constituents, such
as salinity.

\section{Hydrostatic characteristics}\label{hydrostatic-characteristics}

The model is hydrostatic thus vertical accelerations are not taken into
account. The pressure is assumed to vary linearly at each cell along the
vertical direction depending on the density state of the cell, related
to the temperature and salinity. \citep[p.121]{DFlowTechMan}

The horizontal stresses resulting from the vertical stress profile
however are included in the model. This should be further looked into.

\section{Bed level type}\label{bed-level-type}

Certain different bed level types may be specified that are either
piecewise constant (1,2) or piecewise linear (3-6).
