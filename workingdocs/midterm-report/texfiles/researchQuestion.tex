\chapter{Research Question and Hypotheses}\label{research-question}

\section{Research questions}\label{research-questions}

\begin{enumerate}
\def\labelenumi{\arabic{enumi}.}
\item
  How much numerical diffusion and or dispersion does the D-Flow FM
  model produce when simulating a 3D lock-exchange experiment?

  \begin{enumerate}
  \def\labelenumii{\arabic{enumii}.}
  \item
    What is the order of errors produced by D-Flow FM as a result of
    numerical diffusion and/or dispersion?
  \item
    How sensitive is the accuracy of the D-Flow FM model to time, space
    and numerically related parameters?
  \item
    What sort of errors are produced given different parameters?
  \item
    What parameters in the D-Flow FM model have the largest influence on
    errors related to numerical diffusion and dispersion?
  \item
    Are there indicators that predict the occurrence of numerical
    diffusion and dispersion in D-Flow FM?
  \item
    How do the fysics around the mixing layer develop and what influence
    does this have on the accuracy numerical approximation in terms of
    numerical diffusion and -dispersion?
  \item
    How does the frontal wave speed develop compared to solutions that
    might be expected from the characteristic equations?
  \item
    How do internal waves develop at different model parameters and
    settings?
  \item
    What effect do internal waves have on the produced numerical errors?
  \end{enumerate}
\end{enumerate}

\section{Hypotheses}\label{hypotheses}

\begin{enumerate}
\def\labelenumi{\arabic{enumi}.}
\item
  There is significant numerical diffusion and negligible numerical
  dispersion in the D-Flow FM model when performing a lock-exchange
  experiment because the hydrostatic assumptions of the shallow water
  equations, and thus the numerical scheme within D-Flow, can not deal
  with the initial vertical pressure gradient but accounts sufficiently
  for the posed boundary conditions.

  \begin{enumerate}
  \def\labelenumii{\arabic{enumii}.}
  \item
    The order of errors is of 10E-4
  \item
    It is very sensitive to time and space related parameters and can be
    slightly improved by flow model parameters.
  \item
    Mostly numerical diffusion errors are produced except when more
    elaborate advection schemes are applied.
  \item
    The time step size, grid resolution and viscosity have the larges
    influence leverage on the accuracy of the model.
  \item
    Examples of such indicators are the dimensionless Reynolds and
    Froude numbers or the ratio between the net diffusivity coefficient
    in the model and the eddy viscosity terms imposed by D-Flow FM.
  \item
    The physical process is naturally damped by viscosity thus at some
    point the higher-order terms that are included in the truncation
    error of the numerical scheme will become of lesser influence,
    therefore numerical diffusion and dispersion will reduce over time.
    I.e. the shallow water equations will fit the flow better if it is
    more smooth.
  \item
    Riemann invariants may be formed for the flow front if the velocity
    variations far away from the front are negligible and thus the
    frontal wave speed may be considered to constant along a
    characteristic.
  \item
    When viscosity plays a more important role internal wave can be
    expected to be greater. This is at high density differences and at
    high flow velocities.
  \item
    The numerical diffusion is increased a lot by the occurence of a lot
    of internal waves because the approximation is limited in it's
    vertical velocity gradient by the hydrostatic assumption.
  \end{enumerate}
\end{enumerate}
