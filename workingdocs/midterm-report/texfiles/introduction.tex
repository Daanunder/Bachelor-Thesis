\chapter{Introduction}\label{introduction}
Due to temperature and salinity differences density driven currents can
occur. Because important water systems such as the drinking water supply
and large coastal systems in the Netherlands are strongly influenced by
these currents it is important to understand these phenomena. With the
use of D-Flow FM, a software package by Deltares, the mixture of salt-
and fresh water can be numerically approximated. However, a side effect
of these numerical flow models is the occurrence of numerical diffusion
and dispersion caused by characteristics of the numerical discretization
scheme that is used.

\section{Numerical diffusion and
-dispersion}\label{numerical-diffusion-and--dispersion}

Numerical diffusion is sometimes referred to as ``numerical viscosity''
since the associated approximation errors mimic the effect of an
increase in viscosity, i.e.~the solution is overdamped. Since viscous
properties of a fluid decrease the amplitude of the diffusivity rate in
advection-diffusion flow problems. Additionally, numerical dispersion is
related to unrealistic oscillations in an approximation of an
advection-diffusion problem that may occur if stability of the solution
is not ensured or if the observed phenomena are not smooth enough for
the discretized solutions of the problem posed. Because avoiding such
errors requires contrasting measures a quantification of their responses
to certain modelling parameters is desirable. (O'Brien, Hyman, \&
Kaplan, 1950; Zijlema, 2015)

\section{Lock-exchange experiment}\label{lock-exchange-experiment}

In order to quantify the numerical diffusion and dispersion terms in the
D-Flow FM model, and thus get an idea of the model's sensitivity to
certain parameters, a lock-exchange experiment will be set up which
serve as the basis for a sensitivity analysis. A lock-exchange consists
of a closed fluid-tank which can initially be divided into two volumes,
one with a higher density than the other seperated by a impermeable
boundary. At the beginning of the experiment this boundary is abolished
after which the heavier fluid experiences a gravity induced flow and the
lighter fluid (or possibly even gas) experiences a buoyency induced
flow. In this case an experiment with two fluids with an equal initial
waterdepth but different densities, due to a alinity difference, will be
setup.

\section{D-Flow FM}\label{d-flow-fm}

Shallow water equations: Vertical velocity is small compared to
horizontal scale. Vertical pressure nearly hydrostatic. Wave induced
accelerations in the vertical cause negligible effect on the pressure
distribution compared to the wave induced height difference.

Shallow water equations show the water mass density is directly
proportional to bed- and wind friction terms. (p.~35)

\section{Method}\label{method}

\subsection{Problem definition}\label{problem-definition}

After an initial qualitative analysis of the terms and settings found
during an explorative setup of the basic D-Flow FM model, a stable basis
for the sensitivity analysis will be defined. This includes the variable
ranges over which the chosen parameters will be defined and also the
reference model to which the results will be compared. Among others, in
this basic model setup the desired boundary conditions, the lenght and
timescale of the simulation, the constant parameters and general
modelling characteristics will be determined. For further specification
of this basic setup see {[}@ Herafter the creation of such basic model
will be automated so that the data required for the sensitivity analysis
may be generated.

\subsection{Research objective}\label{research-objective}

To be able to refer the sensitivity of numerical errors produced by
D-Flow FM to specific settings and parameters of the model data will be
generated and analyzed. The required data consist of the numerical
errors produced while changing the grid resolution, the timestep size
and parameters related to the flow model. Subsequently the sensitivity S
of a parameter P is defined as the relative change of a state variable
per change of this parameter: S = (δx/x)/(δP/P). \emph{The state
parameters are defined as the difference between the approximation and
the analytical solution at predefined monitoring points, also known as
the accuracy of the approximation.} \textbf{What may we define as the
accuracy of the model in terms of numerical diffusion and dispersion?
What behaviour of the solutions is expected and what do we see in the
produced data?} Prematurely three state parameters are defined; the flow
velocity, the density and wave speed. During the explorative model setup
and the data generation phase, the observed data is analyzed and
compared to certain indicator parameters in order to explain the
observed numerical errors.

\emph{These indicators have previously been found to characterize the
significance of certain modelling parameters (Shin, Dalziel, \& Linden,
2004) of a discretization scheme. Finally the leverage of parameters,
forcing functions and submodules of the model are assessed based on the
observed sensitivity for which a possible explanation in terms of the
indicators and model characteristics will be sought.}

\emph{This document describes the research questions and hypotheses in
the first chapter, subsequently it will describe the research strategy
explained above in further detail in chapter two. Lastly, in the
appendix a planning for this research project can be found.}

\subsection{Basic assumptions}\label{basic-assumptions}

\begin{itemize}
\tightlist
\item
  Hydrostatic pressure

  \begin{itemize}
  \tightlist
  \item
    Depending on the velocity variations seen near the mixing zone this
    is a valid approximation, i.e.~the solution should be smooth
  \item
    Vertical variations in the velocity may still occur as a result of
    the continuity equation
  \end{itemize}
\item
  Resistance can be ignored (Battjes, 2017)

  \begin{itemize}
  \tightlist
  \item
    Damping of wave amplitude should be as little as possible over the
    simulation, in this case the energy conserving assumption is most
    valid
  \item
    Longitudal dispersion and the two dimensional advection-diffusion
    equation should be investigated (Battjes, 2017, p. 202)
  \end{itemize}
\item
  When the above are valid characteristic based methods may be used

  \begin{itemize}
  \tightlist
  \item
    Possibly normalize the wavespeed
  \item
    Forcing terms are density based
  \item
    Resistance is a result of bed- friction and mixing viscosity
  \item
    For two state variables the wave depth (d) and velocity (u) in a
    three dimensional shallow water equation the solution set is a
    volume in the (d, x, y, t)-space and (u, x, y, t)-space
  \end{itemize}
\item
  By examening the flow based on the above assumptions numerical errors
  may be be recognized:

  \begin{itemize}
  \tightlist
  \item
    The hydrostatic assumption is least valid at t=0 because of the
    hydrostatic diferences between the two sides of the front thus this
    is where numerical diffusion may be at its extreme
  \item
    Resistance may be of significant influence at the mixing layer,
    especially where internal waves are formed, thus numerical diffusion
    may be significant
  \item
    Where the propagation of the disturbance does not follow the
    expected propagation in the s-t plane (probably is too damped)
    diffusion may be recognized
  \end{itemize}
\end{itemize}

\section{Literature}\label{literature}

\textbf{Insert results found in literature and related research}

\hypertarget{refs}{}
\hypertarget{ref-Battjes2017}{}
Battjes, J. A. (2017). \emph{Unsteady flow in open channels}. Retrieved
from
\url{https://www.ebook.de/de/product/27503143/jurjen_a_battjes_unsteady_flow_in_open_channels.html}

\hypertarget{ref-obrien_study_1950}{}
O'Brien, G. G., Hyman, M. A., \& Kaplan, S. (1950). A Study of the
Numerical Solution of Partial Differential Equations. \emph{Journal of
Mathematics and Physics}, \emph{29}(1-4), 223--251.
\url{https://doi.org/10.1002/sapm1950291223}

\hypertarget{ref-Shin2004}{}
Shin, J., Dalziel, S., \& Linden, P. (2004). Gravity currents produced
by lock exchange. \emph{Journal of Fluid Mechanics}, \emph{521}, 1--34.

\hypertarget{ref-zijlema_computational_2015}{}
Zijlema, M. (2015). Computational Modelling of Flow and Transport.
\emph{Collegedictaat Cie4340}. Retrieved from
\url{https://repository.tudelft.nl/islandora/object/uuid\%3A5e7dac47-0159-4af3-b1d8-32d37d2a8406}
