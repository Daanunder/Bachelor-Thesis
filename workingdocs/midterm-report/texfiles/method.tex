\chapter{Method}\label{method}

Steps to be incorporated in planning: 1. Get familiar with model *
Tutorials 2. Run initial lock-exchange experiment * Setup general
settings * Setup basic geometry * Setup physics \& boundary conditions *
Setup computational grid * Setup monitoring points 3. Perform analysis
on obtained results * Perform qualitative analysis 4. Refine reference
model * Refine model based on qualitative analysis * Compute Riemann
invariants * Compute indicator variables * Refine again until desired
reference model is obtained - \textbf{What parameter interval and
numerical conditions may be applicable from the section data generation}
5. Automate standard post processing steps * Script standard data
analysis on results (normalized) * Script characteristic (to be expected
results) * Script indicator parameters (graphs e.g.) 6. Perform data
generation 7. Perform data analysis

\section{D-Flow FM setup}\label{d-flow-fm-setup}

In order to provide a solid basis for the sensitivity analysis the basic
model has to be applicable to the research objective and sufficiently
robust to enable all parameters to be explored. To ensure a working
model first most default settings will be used to set up the
lock-exchange model, \emph{this will already require a minimal
definition file where general settings, physics, geometry, flow and
numerical parameters are set.} Finally, monitoring points have to be set
up at logically sound locations within the basic models in order for the
subsequent qualitative analysis and post-processing to be relevant.
These monitoring points will be formed at locations where anomaliies in
the approximation may be found, thus where numerical errors may be
observed in extreme forms. This will be around the wave front, the
boundary conditions and near the mixing layer where internal waves may
form. It can be discuessed whether these points are stationary or
whether it is desirable that this frame of reference moves with a
certain speed, e.g.~the wavespeed of the front.

\subsection{Initial qualitative
analysis}\label{initial-qualitative-analysis}

For an inital qualitative analysis the state variables and indicator
parameters are evaluated at points where they either have an extreme
value or where characteristic values of the solution can be expected.
This thus depends on the Riemann invariants formed for the particular
situation, taking friction and gravitational forcing into account.

\subsection{Refinement of the basic
model}\label{refinement-of-the-basic-model}

Thereafter possibly more realistic physical properties, boundary
conditions, a more relevant geometry and specifics of the flow model
such as bed friction and the will be defined based on the qualitative
analysis of the produced results. This definition of the reference model
should be done with the research objective in mind \emph{thus possibly
requires setting up three different models for three different mesh
grids in order to enable stable testing, at the cost of consistency
between models. These trade-offs with respect to the research
applicability and generalizability will have to be made.} Refinement of
the model may be done by adjusting the parameters below, it should be
noted however that some of these are to remain constant:

\begin{itemize}
\tightlist
\item
  Viscosity
\item
  Initial salinity / density difference
\item
  Bed friction
\item
  Lenght scale
\item
  Time scale
\item
  Height / water depth
\item
  Grid generation \textbf{Near, intermediate and far field
  characteristics of transport flows}
\item
  Monitoring points \textbf{Can this be a moving frame of reference,
  e.g.~based on the expected wavespeed}
\item
  Entrainment (mixing terms)
\item
  Advection scheme
\end{itemize}

\subsection{Post processing setup}\label{post-processing-setup}

Further on the first post-processing of the generated results by the
refined model will be performed, in this way the results of the basic
model can be processed and evaluated with respect to multiple indicators
and the expected flow characteristics. These steps are also scripted so
they can easily be used again, besides being selected on relevance.

\begin{itemize}
\tightlist
\item
  What may be expected based on the characteristic method?
\item
  How can we normalize the obtained results?
\item
  What indicator parametres are of relevance?
\end{itemize}

\section{Data generation}\label{data-generation}

At the start of the data generation the research strategy has almost
been fully developed. But some important questions remain.

\begin{itemize}
\tightlist
\item
  Which parameters are interesting to change?
\item
  What range of values should be investigated?
\item
  How many experiments can be run?
\end{itemize}

For this a few evaluations can be made with respect to the numerical
model and the governing equations:

\begin{itemize}
\tightlist
\item
  Interval analyis
\item
  Numerical conditions

  \begin{itemize}
  \tightlist
  \item
    Consistency, stability \& convergence (Basic Numerical maths)
  \item
    Von Neumann condition (Amplification factor)
  \item
    CFL condition (Positive numerical diffusion, domain of influence)
  \item
    Computational stability (Domain of influence)
  \item
    Heuristic stability
  \item
    Monotonicity
  \end{itemize}
\item
  If significant numerical dispersion is observed a spectral analysis
  may be performed as proposed by Ruano, Báez Vidal, Trias, \& Rigola
  (2019)
\end{itemize}

\section{Data analysis}\label{data-analysis}

During the data analysis the results of all the post processing of the
different performed experiments are analyzed and a relation between the
sensitiviy of the observed numerical errors to certain the change in
certain parameters and computeded indicator parameters are sought for
based on the model used and the characteristics of the experiment.

\begin{itemize}
\tightlist
\item
  What indicator parameters are of significance and why?
\item
  Can the observed sensitivity of the numerical errors in the produced
  results to a parameter in the indicator parameters be related to the
  computed indicator parameters?
\end{itemize}

\section{D-Flow FM setup}\label{d-flow-fm-setup-1}

\section{Initial model}\label{initial-model}

\subsection{General settings}\label{general-settings}

\subsection{Physiscs}\label{physiscs}

\subsection{Geometry}\label{geometry}

\subsection{Flow}\label{flow}

\subsection{Numerics}\label{numerics}

\subsection{Meshgrid}\label{meshgrid}

\subsection{Monitoring points}\label{monitoring-points}

\section{Qualitative analysis}\label{qualitative-analysis}

\subsection{Extreme values}\label{extreme-values}

\subsection{Characteristic locations}\label{characteristic-locations}

\section{Post Processing}\label{post-processing}

\subsection{Normalized results}\label{normalized-results}

\subsection{Riemann invariants}\label{riemann-invariants}

\subsection*{Indicator parameters}\label{indicator-parameters}
\addcontentsline{toc}{subsection}{Indicator parameters}

\hypertarget{refs}{}
\hypertarget{ref-Ruano2019}{}
Ruano, J., Báez Vidal, A., Trias, F., \& Rigola, J. (2019). \emph{A
general method to compute numerical dispersion errors and its
application to stretched meshes}.
